\documentclass[12pt]{article}
\usepackage[dvipsnames]{xcolor}
\usepackage{hyperref, pagecolor, mdframed }
\usepackage{graphicx, amsmath, latexsym, amsfonts, amssymb, amsthm,
amscd, geometry, xspace, enumerate, mathtools}
\usepackage{tikz}
\newcommand{\F}{\mathcal{F}}
\newcommand{\Topo}{\mathfrak{T}\mathfrak{o}\mathfrak{p}}
\newcommand{\G}{\mathcal{G}}


\definecolor{wgrey}{RGB}{50,28,77}
\hypersetup{
    colorlinks=true,
    linkcolor=blue,
    urlcolor=red
}
\title{Faisceaux, notes 1}
\date{7 juillet 2023}
\begin{document}
\tableofcontents
\maketitle

\section{Première défs et props}
Ducoup, $X$ un e.t. et $\Topo(X)$ la catégorie des ouverts de $X$ où les morphismes sont les inclusions.

\textbf{\color{wgrey} Def} : Un faisceaux $\F$ sur $X$ est un foncteur contravariant de $\Topo$ dans $Ab/Ring/etc.$.

\end{document}