\documentclass[12pt]{article}
\usepackage[dvipsnames]{xcolor}
\usepackage{hyperref, pagecolor, mdframed }
\usepackage{graphicx, amsmath, latexsym, amsfonts, amssymb, amsthm,
amscd, geometry, xspace, enumerate, mathtools}
\usepackage{tikz}
\usepackage{tabularx, array}

\setlength{\oddsidemargin}{-10mm}
\setlength{\evensidemargin}{5mm}
\setlength{\textwidth}{175mm}
\setlength{\headsep}{0mm}
\setlength{\topmargin}{0mm}
\setlength{\textheight}{220mm}
\setlength\parindent{24pt}

\theoremstyle{plain}
\newtheorem{thm}[subsubsection]{Th\'eor\`eme}

\newcommand{\fdiv}{\textrm{div}}
\newcommand{\Z}{\mathbb{Z}}
\newcommand{\Q}{\mathbb{Q}}
\newcommand{\algK}{\overline{K}}
\newcommand{\algF}{\overline{\mathbb{F}}}
\newcommand{\Pic}{\textrm{Pic}}
\newcommand{\Hom}{\textrm{Hom}}
\newcommand{\End}{\textrm{End}}
\newcommand{\Disc}{\textrm{Disc}}
\newcommand{\Det}{\textrm{Det}}
\newcommand{\Tr}{\textrm{Tr}}
\newcommand{\Or}{\mathcal{O}}
\newcommand{\OK}{\mathcal{O}_{K}}
\newcommand{\OL}{\mathcal{O}_{L}}
\newcommand{\C}{\mathbb{C}}
\newcommand{\ai}{\mathfrak{a}}
\newcommand{\bi}{\mathfrak{b}}
\newcommand{\w}{\omega}
\newcommand{\gr}{\color{Sepia}}
\newcommand{\rg}{\color{Red}}
\hypersetup{
    colorlinks=true,
    linkcolor=blue,
    urlcolor=Green,
    filecolor=RoyalPurple
}

\newcolumntype{M}[1]{>{\raggedright}m{#1}}

\definecolor{wgrey}{RGB}{148, 38, 55}

\title{NUCOMP}
\date{15 septembre 2023}
\begin{document}
\maketitle

\begin{center}
\begin{tabular}{M{4 cm} M{4 cm} M{4 cm}}
    Espace & éléments & base \\
    \tabularnewline
    $\bigwedge^r V$ & $\sum_{i=1}^{r} a_i \bigwedge_j^i e_{i_j}$  & $(e_{i_1}\wedge...\wedge e_{i_r})$\\
    \tabularnewline 
    $V$&$\sum_i a_ie_i$ &$(e_1,...,e_r)$  \\
    \tabularnewline
    & \tabularnewline
  
    
\end{tabular}
\end{center}

On regarde $S_r(V)$ l'ensemble des familles libres de $V$ de dim $r$. Et on a :
\begin{align*}
    \phi~:~S_r(V)&\rightarrow \mathcal{P}(\bigwedge^r V)\\
    (f_1,...,f_r)&\mapsto f_1\wedge ...\wedge f_r\\
\end{align*}

(On a une action de $GL_r(V)$ qui passe au quotient ?)




\end{document}