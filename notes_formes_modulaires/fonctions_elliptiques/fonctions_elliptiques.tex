
\documentclass[12pt]{article}
\usepackage[dvipsnames]{xcolor}
\usepackage{hyperref, pagecolor, mdframed }
\usepackage{graphicx, amsmath, latexsym, amsfonts, amssymb, amsthm,
amscd, geometry, xspace, enumerate, mathtools}
\usepackage{tikz}

\theoremstyle{plain}
\newtheorem{thm}[subsubsection]{Th\'eor\`eme}
\newtheorem{lem}[subsubsection]{Lemme}
\newtheorem{prop}[subsubsection]{Proposition}
\newtheorem{propr}[subsubsection]{Propri\'et\'e}
\newtheorem{cor}[subsubsection]{Corollaire}
\newtheorem{intro}[section]{Introduction}
\newtheorem{thm2}[subsection]{Th\'eor\`eme}
\newtheorem{lem2}[subsection]{Lemme}
\newtheorem{prop2}[subsection]{Proposition}
\newtheorem{propr2}[subsection]{Propri\'et\'e}
\newtheorem{cor2}[subsection]{Corollaire}

\theoremstyle{definition}
\newtheorem{defn}[subsubsection]{D\'efinition}
\newtheorem{rmq}[subsubsection]{Remarque}
\newtheorem{conj}[subsubsection]{Conjecture}
\newtheorem{exmp}[subsubsection]{Exemples}
\newtheorem{quest}[subsubsection]{Exercices}
\newtheorem{defn2}[subsection]{D\'efinition}
\newtheorem{rmq2}[subsection]{Remarque}
\newtheorem{conj2}[subsection]{Conjecture}
\newtheorem{exmp2}[subsection]{Exemples}
\newtheorem{quest2}[subsection]{Exercices}

\theoremstyle{remark}
\newtheorem{rem}{Remarque}
\newtheorem{note}{Note}

\newcommand{\fdiv}{\textrm{div}}
\newcommand{\Z}{\mathbb{Z}}
\newcommand{\Q}{\mathbb{Q}}
\newcommand{\K}{\mathbb{K}}
\newcommand{\Proj}{\mathbb{P}}
\newcommand{\algK}{\overline{K}}
\newcommand{\algF}{\overline{\mathbb{F}}}
\newcommand{\Pic}{\textrm{Pic}}
\newcommand{\Hom}{\textrm{Hom}}
\newcommand{\End}{\textrm{End}}
\newcommand{\C}{\mathbb{C}}
\newcommand{\w}{\omega}
\newcommand{\h}{\mathfrak{h}}
\newcommand{\La}{\mathcal{L}}
\newcommand{\F}{\mathcal{F}}
\hypersetup{
    colorlinks=true,
    linkcolor=blue,
    urlcolor=Green,
    filecolor=RoyalPurple
}

\definecolor{wgrey}{RGB}{148, 38, 55}
\title{Fonctions elliptiques}
\date{13 aout 2023}
\begin{document}
\tableofcontents
\maketitle

Quelques rappels ducoup sur la fct $\wp$ de weierstrass.
\section{fonctions elliptiques et multiplication complexe}
Là le but c'est de lier corps quadratiques imaginaires et courbes elliptiques.\\
En gros ce sera les fonctions régulières sur la courbe je crois. Une fonction est elliptique pour un réseau $L$ si meromorphe sur $\C$ et périodique sur 
$L$. 

\begin{itemize}
    \item En gros si $L=[\w_1, \w_2]$ et $f$ est elliptique pour $L$, $f$ est holomorphe sur $\C\backslash L$ et $\forall z~f(z+\w_1)=f(z+\w_2)=f(z)$.
\end{itemize}

\subsection{fonction rho de Weierstrass}
Via le Cox p.182. On pose $\wp(z;L)=~\frac{1}{z^2}~+~\sum_{\w\in L\setminus{0}}\left(\frac{1}{z^2}-\frac{1}{\w^2}\right)$. D'abord la convergence, l'holomorphie tout ça c'est donné par : 
\begin{itemize}
    \item $G_r(L):=\sum_{L\setminus{0}}\frac{1}{\w^r}$ converge absolument pour $r>2$. Via $$\lvert G_r(L)\rvert\leq\frac{1}{M^r}\sum_{m,n} \frac{1}{\lvert m^2+n^2\rvert^r}$$ et en comparant
à l'intégrale de Riemann convergente corresp :$\qed$
    \item Ensuite ça permet de majorer le terme de la $\wp$ par $G_3$ et conclure.
    \item $\wp$ est paire.
    \item $\wp$ est périodique : $\wp '$ l'est trivialement ducoup écrire $\wp(\w_1/2)=\wp(-\w_1/2)+C$ puis $C=0$.
\end{itemize}

On cherche la relation qui donne la courbe elliptique : l'idée est souvent la même, 
Supprimer les poles puis trouver des 0. Comme pour $x<1$ : $\frac{1}{(1-x)^2}=1+\sum_{n\geq1} (n+1)x^n$. On a :
\begin{itemize}
    \item Autour de $0$ : $\wp(z)=\frac{1}{z^2} + \sum_{n=1} (2n+1)G_{2n+2}z^{2n}$, le $2$ vient de la parité.
    \item Autour de $0$ : $\wp '(z)=\frac{-2}{z^3} + \sum_{n=1}(2n+1)(2n)G_{2n+2}z^{2n-1}$.
    \item D'où en comparant les coeffs de $\wp^3$ et $(\wp ')^2$ et en posant $g_2(L)=60G_4(L)$ et $g_3(L)=140G_6(L)$ on obtient que:
    $$F(z)=\wp '(z)^2-4\wp(z)^3+g_2\wp(z)+g_3$$
s'annule en $0$ et donc sur tout $L$ mais par periodicité reste bornée donc constante. i.e. $$\wp '(z)^2=4\wp(z)^3-g_2\wp(z)-g_3$$
    \item Enfin : $\wp(z)=\wp(w)~\leftrightarrow~z\equiv w~mod~L$ (utiliser la formule de weierstrass, voir poly).
    \item D'où $\wp '(z)=0\leftrightarrow 2z\in L$, c'est les points d'ordre $2$.
    \item Formule de l'addition : $\wp(z+w)=-\wp(z)-\wp(w)+\frac{1}{4}(\frac{\wp '(z)-\wp '(w)}{\wp(z)-\wp(w)})^2$. A nouveau y faut écrire les
    développements de Laurent et supprimer les pôles. (cas $P\ne-Q$ : l'hospital, cas $2P\in L$ a faire).  
\end{itemize}

\subsection{j-invariant}
On pose :
\begin{itemize}
    \item $\Delta(L)=\textrm{Disc}(4x^3-g_2(L)x-g_3(L))=g_2(L)^3-27g_3(L)^2$.
    \item $j(L)=1728\frac{g_2(L)^3}{\Delta(L)}$.
\end{itemize}

La dèf de $j$ est valable car $\Delta$ est nulle ssi la courbe est singulière sauf que, les points d'ordres $2$ sont on a vu :$$\w_1/2,~\w_2/2~,~\frac{\w_1+\w_2}{2}$$
Le point important est que :\begin{itemize}
    \item $j(L)=j(L')\leftrightarrow L'=\lambda L$.
\end{itemize}

En fait on regarde ce qu'y se passe quand on a seulement $g_2(L')=\lambda^{-4}g_2(L)$ et $g_3(L')=\lambda^{-6}g_3(L)$. Comme 
\begin{itemize}
    \item Les coeffs du développements de Laurent $\wp(z)=\frac{1}{z^2} + \sum_{n=1} (2n+1)G_{2n+2}z^{2n}$ sont des polynomes
    à coefficients rationnels indép de $L$ en les $g_2,~g_3$. (dériver l'équa diff et comparer les coeffs)
\end{itemize}

Ducoup $\wp(z;L')=\wp(z;\lambda L)$ puis $L'=\lambda L$ car les pôles de $\wp$ sont le réseau associé !

\subsection{Multiplication complexe}
Bon la ca se corse, y faut parler de corps quadratiques et d'ordres. Puis de corps de classe de Hilbert. Du coup,
je fais un autre Latex qui se concentre dessus.


\end{document}