\documentclass[12pt]{article}
\usepackage[dvipsnames]{xcolor}
\usepackage{hyperref, pagecolor, mdframed }
\usepackage{graphicx, amsmath, latexsym, amsfonts, amssymb, amsthm,
amscd, geometry, xspace, enumerate, mathtools}
\usepackage{tikz}

\theoremstyle{plain}
\newtheorem{thm}[subsubsection]{Th\'eor\`eme}

\newcommand{\fdiv}{\textrm{div}}
\newcommand{\Z}{\mathbb{Z}}
\newcommand{\Q}{\mathbb{Q}}
\newcommand{\algK}{\overline{K}}
\newcommand{\algF}{\overline{\mathbb{F}}}
\newcommand{\Pic}{\textrm{Pic}}
\newcommand{\Hom}{\textrm{Hom}}
\newcommand{\End}{\textrm{End}}
\newcommand{\Disc}{\textrm{Disc}}
\newcommand{\Det}{\textrm{Det}}
\newcommand{\Tr}{\textrm{Tr}}
\newcommand{\Or}{\mathcal{O}}
\newcommand{\OK}{\mathcal{O}_{K}}
\newcommand{\OL}{\mathcal{O}_{L}}
\newcommand{\C}{\mathbb{C}}
\newcommand{\ai}{\mathfrak{a}}
\newcommand{\bi}{\mathfrak{b}}
\newcommand{\w}{\omega}
\newcommand{\gr}{\color{Sepia}}
\newcommand{\rg}{\color{Red}}
\hypersetup{
    colorlinks=true,
    linkcolor=blue,
    urlcolor=Green,
    filecolor=RoyalPurple
}

\definecolor{wgrey}{RGB}{148, 38, 55}

\title{Valuations/Complete fields}
\date{21 septembre 2023}

\begin{document}
\maketitle
\tableofcontents

\section{les nombres p-adiques}
On définit $\Z_p$ par
\begin{enumerate}
    \item $\underleftarrow{lim} \Z/p^\Z$ muni de la topologie induite du produit des topologies discrètes, la structure d'anneau est alors évidente.
La structure un peu moins.
    \item L'espace des séries formelles $\sum_{i=0}^{\infty} a_ip^i$ ou la l'injection de $\Z$ est donnée par l'écriture en base $p$
La complétude est plus claire mais la structure d'anneau un peu moins.
\end{enumerate}
$\Q_p$ est donné par les corps de fraction. Qui sont en fait isomorphes à la complétion de $\Q$ en $v_p$.
L'utilité vient en partie de la réduction :
\begin{itemize}
    \item Etant donné une équation $F(x_1,...,x_n)=0$.
    \item On la réduit modulo $m$ puis par le CRT modulo $p^{\nu}$ $\forall \nu$.
\end{itemize}

Et en fait :
\begin{itemize}
    \item $\forall~\nu$ il existe une solution mod $p^{\nu}$ ssi il en existe une primitive dans $\Q_p$, ou une dans $\Z_p$(equivalent).
    \item En fait il suffit d'un $\nu$ par le lemme de Hensel.
\end{itemize}




\end{document}