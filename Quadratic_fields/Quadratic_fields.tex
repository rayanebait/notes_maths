\documentclass[12pt]{article}
\usepackage[dvipsnames]{xcolor}
\usepackage{hyperref, pagecolor, mdframed }
\usepackage{graphicx, amsmath, latexsym, amsfonts, amssymb, amsthm,
amscd, geometry, xspace, enumerate, mathtools}
\usepackage{tikz}

\theoremstyle{plain}
\newtheorem{thm}[subsubsection]{Th\'eor\`eme}

\newcommand{\fdiv}{\textrm{div}}
\newcommand{\Z}{\mathbb{Z}}
\newcommand{\Q}{\mathbb{Q}}
\newcommand{\algK}{\overline{K}}
\newcommand{\algF}{\overline{\mathbb{F}}}
\newcommand{\Pic}{\textrm{Pic}}
\newcommand{\Hom}{\textrm{Hom}}
\newcommand{\End}{\textrm{End}}
\newcommand{\Disc}{\textrm{Disc}}
\newcommand{\Det}{\textrm{Det}}
\newcommand{\Tr}{\textrm{Tr}}
\newcommand{\Or}{\mathcal{O}}
\newcommand{\OK}{\mathcal{O}_{K}}
\newcommand{\OL}{\mathcal{O}_{L}}
\newcommand{\C}{\mathbb{C}}
\newcommand{\ai}{\mathfrak{a}}
\newcommand{\bi}{\mathfrak{b}}
\newcommand{\w}{\omega}
\newcommand{\gr}{\color{Sepia}}
\newcommand{\rg}{\color{Red}}
\hypersetup{
    colorlinks=true,
    linkcolor=blue,
    urlcolor=Green,
    filecolor=RoyalPurple
}

\definecolor{wgrey}{RGB}{148, 38, 55}

\title{Quadratic fields}
\date{24 aout 2023}
\begin{document}
\tableofcontents
\maketitle

Le but est de pouvoir lire grosso modo \href{file:///C:/Users/33771/Desktop/articles/Hilbert_class_poly_algo.pdf}{ça} 
pour \href{file:///C:/Users/33771/Desktop/modular_polynomials_via_isogeny_volcanoes.pdf}{ça}.
Je suis le traitement de Cox.

\section{Bases}
Une extension quadratique de $\Q$ est tjr de la forme :

\begin{itemize}
    \item $\Q(\sqrt{n})$, $n$ sans facteurs carrés. (Facile)
    \item Ensuite le discriminant usuel $D_{K/\Q}=\Det(\Tr(x_ix_j))$ : Si $n\equiv 1~mod~4$ alors $D=D_{K/\Q}=n$
    si $n\equiv 2,3~mod~4$ alors $D=-4n$. (Penser à $X^2-X+(1-n)/4$)
    \item Aussi $\frac{1+\sqrt{n}}{2}$ engendre $\OK$ dans le premier cas $\sqrt{n}$ dans le second.
    \item La ramification est simple aussi, avec le lemme chinois on prédit :\\
            $\left(\frac{d_K}{p}\right)=-1,~0,~1$ donne $p$ inerte, $p$ ramifié, $p$ split.
\end{itemize}


\section{Formes quadratiques binaires}
\subsection{Binary quadratic forms}
Bon là c'est plus pour moi, je note des défs pour avoir des trucs à écrire et travailler dessus.
On regarde des formes quadratiques entières en deux variables, $f(x,y)=ax^2+bxy+cy^2$, a b et c $\in\Z$.\\

\noindent \textbf{\rg Definitions : }
\begin{itemize}
    \item $f$ est primitive si les coeffs sont premiers entre eux.
    \item $f$ représente un entier $m$ si $\exists~x,y\in\Z$ tq $f(x,y)=m$
    \item $f$ représente proprement $m$ si $x,y$ sont premiers entre eux.
    \item $f$ et $g$ sont équivalentes si il existe un $M=\begin{pmatrix}
        p & r\\
        q & s
    \end{pmatrix}\in GL_2(\Z)$ tq $f=g\circ M$.
    \item L'equivalence est propre si $\Det(M)=1$ impropre sinon.
    \item à noter : si $ps-qr=1$, alors $f(px+qy,~rx+sy)=f(p,r)x^2+(2apq+bps+brq+2crs)+f(q,s)y^2$.
    \item ducoup $f$ représente proprement $m$ ssi $f$ est proprement eq à $mx^2+Bxy+Cy^2$.
\end{itemize}
La plupart des notions sont invariantes par équivalences. En particulier, le fait d'être primitive, positive et définie. (fait le c'est simple)

\subsubsection{Discriminant}
On note $D=b^2-4ac$ le discriminant de $f$. En particulier, $x^2+ny^2$ a discriminant $-4n$.\\

\noindent \textbf{\rg Représentants et discriminants :}
\begin{itemize}
    \item $D_f=\Det(M)^2D_{f\circ M}$. D'où deux formes \textbf{\gr équivalentes} ont même \textbf{\gr discriminant}.
    \item $D_f\equiv 0,1~mod~4$ car $D_f\equiv b^2~mod~4$.
    \item $4af(x,y)=(2ax+by)^2-D_fy^2$ d'où $f$ est \textbf{\gr définie} si $D_f<0$ \textbf{\gr indéfinie} sinon.
    \item Si $m$ est impair premier à $D\equiv0,1~mod~4$ alors il existe $f$ \textbf{\gr primitive} qui représente \textbf{\gr proprement} $m$ ssi $\left(\frac{D}{m}\right)=1$.
    ($m$ impair comme ça on peut supp $D\equiv b^2~mod~4$)
    \item En particulier $\left(\frac{-n}{p}\right)=1$ ssi $p$ est représentée par une \textbf{\gr forme primitive} de $D=-4n$. ($\left(\frac{-n}{p}\right)=\left(\frac{-4n}{p}\right)$)
\end{itemize}

Pour résumer, si par exemple on cherche à représenter $p$ via $x^2+ny^2$ alors on doit avoir $\left(\frac{-n}{p}\right)=1$. Par contre
$\left(\frac{-n}{p}\right)=1$ a une solution ne précise pas la forme de discriminant $-4n$ qui représente $p$. Il faut donc déterminer 
des représentants des classes d'équivalences.

\subsubsection{Reduced form}
On se ramène au cas des formes \textbf{\gr définies positives}($-D_f,~a\geq 0$). (cas de $x^2+ny^2$)
Une forme $f$ est réduite si $|b|\leq a\leq c$ et $b\geq 0$ si $|b|=a$ ou $a=c$.\\

\noindent\textbf{\rg Premiere réduction :}
\begin{itemize}
    \item Toute forme \textbf{\gr primitive}, \textbf{\gr définie}, \textbf{\gr positive} est \textbf{\gr proprement équivalente} à une forme \textbf{\gr réduite}.
\end{itemize}

A noter de la preuve [p.25, Cox], \begin{description}
    \item[Step 1] On prend $b$ minimale dans la classe d'équivalence propre, alors $\mid b\mid\leq a,c$.
    \item[Step 2] Une telle forme est proprement équivalente à une forme réduite. 
    \item[Step 3] L'étape de l'unicité est intéressante : Cas général, $\mid b\mid<a<c$ alors $$f(x,y)\geq (a-\mid b\mid+c)\textrm{min}(x^2,y^2)$$
    d'ou en regardant les valeurs de $x,y$ on remarque que $a$ est la valeur minimale non nulle de $f$, $c$ la suivante et $(a-\mid b\mid+c)$ la suivante !
    L'\textbf{\gr unicité} en découle.
\end{description}

\noindent En particulier avoir les formes réduites c'est avoir les 3 valeurs minimales de toutes les formes de la classe. Enfin :
Si $f$ est réduite alors : $-D=4ac-b^2\geq 4a^2-a^2=3a^2$ d'où $$a\leq \sqrt{-D/3}$$
D'où un nombre fini de $a$ possible pour une forme réduite de discriminant $D$.
\begin{description}
    \item[\gr Nombre de classes] Comme $\mid b\mid\leq a$ et $D=b^2-4ac$ on a un nombre fini de formes réduites possible de discriminant $D$. Et on note 
    $h(D)$ le nombre de classe d'équivalences propre de discriminant $D$.
\end{description}

On a donc :
\begin{description}
    \item[\rg Première réduction] $\left(\frac{-n}{p}\right)=1$ ssi $p$ est representé par une des $h(-4n)$ formes réduites.
\end{description}

\textbf{Dû au fait que $\left(\frac{-n}{p}\right)=\left(\frac{-4n}{p}\right)$ et la dernière remarque de la dernière section.}

\begin{description}
    \item[\rg Généralisation] Si $\chi$ est le caractère quadratique de $(\Z/D\Z)^{\times}$ alors pour $p\ne D$, $[p]\in ker(\chi)$ ssi $p$ est 
representé par une des $h(D)$ formes réduites de discriminant $D$.
\end{description}

\textbf{Le point est que cette fois, on regarde que la classe de $p$ mod $D$.}
Les théorèmes sont surpuissants là, par exemple on peut résoudre le problème pour $n=2,3,7$ facile. Parce qu'en fait :
\begin{itemize}
    \item $x^2+y^2$ est la seule forme réduite de discriminant $-4$. ($\sqrt{4/3}<2$)
    \item $x^2+2y^2$ est la seule forme réduite de discriminant $-8$. ($\sqrt{8/3}<2$)
    \item $x^2+3y^2$ est la seule forme réduite de discriminant $-12$. ($\sqrt{4}=2> a$) ($\mid b\mid < a$)
    \item $x^2+7y^2$ est la seule forme réduite de discriminant $-28$. ($\sqrt{28/3}\approx 3 > a$) (On vérifie directement que $a=2$ n'est pas poss.) 
\end{itemize}

D'où via le théorème : 
\begin{description}
    \item[Via ker$(\chi_4)\cap\{\overline{p}\mid p\ne 4,~\left(\frac{4}{p}\right)=1\}=\{1\}$ :] On a, $p=x^2+y^2\leftrightarrow p\equiv 1~mod~4$.
    \item[Via ker$(\chi_8)\cap\{\overline{p}\mid p\ne 8,~\left(\frac{8}{p}\right)=1\}=\{1, 3\}$ :] On a, $p=x^2+y^2\leftrightarrow p\equiv 1,3~mod~8$.
    \item[Via ker$(\chi_{12})\cap\{\overline{p}\mid p\ne 12,~\left(\frac{12}{p}\right)=1\}=\{1, 7\}$ :] Dans cette étape comme $2$ est un facteur d'exposant pair, on peut réduire à $1~mod~3$ ou $p=3$. On a, $$p=x^2+3y^2\leftrightarrow p\equiv 1~mod~3$$
    \item[Via ker$(\chi_{28})\cap\{\overline{p}\mid p\ne 28,~\left(\frac{24}{p}\right)=1\}=\{1,9,11,15,23,25\}$ :] $$p=x^2+7y^2\leftrightarrow p\equiv 1,9,11,15,23,25~mod~28$$.
\end{description}

L'idée marche quand y'a qu'une seule forme réduite mais c'est que rarement le cas. En fait :
\begin{itemize}
    \item (Landau) $h(-4n)=1$ ssi $1,2,3,4,7$.
\end{itemize}
\newpage
\subsection{elementary genus theory}

\textbf{\rg Seconde réduction :} Ducoup pour l'instant ker$(\chi)$ est l'ensemble des premiers représentés par UNE des classes de disc $D$.
Maintenant on réduit à des un sous groupe, i.e. on a des formes qui représentent des coset d'un sous groupe.

\begin{description}
    \item[\rg Forme principale, $D\equiv0~mod~4$] Def comme $x^2-\frac{D}{4}y^2$.
    \item[\rg Forme principale, $D\equiv1~mod~4$] Def comme $x^2+xy+\frac{1-D}{4}y^2$.
\end{description}

Maintenant la réduction.
\begin{description}
    \item[Déjà,] Les valeurs dans $\Z/D\Z$ représentées par la forme principale forment un sous groupe $H\leq$ker$(\chi)$! 
    \item[En plus,] Toute forme de discriminant $D$ représentent un élément de ker$(\chi)/H$.
\end{description}

La preuve de la premiere partie est assez simple :
\begin{description}
    \item[Premiere assertion :] $(x^2+ny^2)(z^2+nw^2)=(xz\pm nyw)^2+n(xw\pm yz)^2$ montre la stabilité dans le premier cas ($D=-4n$), et de $4(x^2+xy+\frac{1-D}{4}y^2)\equiv (2x+y)^2~mod~D$
On déduit le second. 
    \item[Seconde assertion :] Pour montrer que les valeurs représentées sont dans un coset c'est malin : y suffit de l'existence d'une valeur representée proprement première à $D$. Disons $a$. 
    Alors déjà on peut supposer que $f(x,y)=ax^2+bxy+cy^2$ on est passé d'une valeur au cas général ! mtn faut se ramener à la forme principale : On a dans le cas $D=-4n$ : $af(x,y)=(ax+b/2y)^2+ny^2$ ($b$ est pair). D'où
les valeurs sont dans $a^{-1}H$. Le fait que toute les valeurs soient représentées c'est : Si $ac\in H$ alors $ac=z^2+nw^2~mod~D$ et via l'équation précédente on résout $f(x,y)\equiv c~mod~D$.
\end{description}

Le corollaire intéressant pour la forme principale c'est que : 
\begin{itemize}
    \item ($D=-4n$) Si $p$ est représenté par une forme du genre principal alors la forme principale représente un entier congru 
    à $p$ mod $D$. D'où $p$ est représenté par une forme de genre principal ssi il existe $\beta$ t.q $$p\equiv\beta^2,\beta^2+n~mod~4n$$
\end{itemize}

En particulier, si le genre principal est formé d'une seule classe on peut résoudre directement $p=x^2+ny^2$. C'est le cas pour :
$n=5, 6, 10, 13, 15, 21, 22, 30$ par exemple. (Exo : résoudre le problème.)

Deux continuation : on peut déterminer des infos sur le groupe, on peut chercher à représenter des produits de premier.
C'est vraiment joli, c'est la suite.  
\newpage
\subsection{Genus theory}
L'idée va être de généraliser et simplifier ce genre d'équivalence : (il existe une "composition")
\begin{itemize}
    \item $(2x^2+2xy+3y^2)(2z^2+2zw+3w^2)=(2xz+xw+yz+3yw)^2+5(xw-yz)^2$
    \item Si et seulement si il existe $p$ tel que $(2x^2+2xy+3y^2)$ représente p. Il existe q tel que $(2z^2+2zw+3w^2)$ représente $q$ et $x^2+5y^2$ représente $pq$.

\end{itemize}

Pour $D<0$ un discriminant. On appelle $C(D)$ l'ensemble des classes de formes de discriminant $D$ modulo équivalence \textbf{propre}.
On écrit aussi $f(x,y)=ax^2+bxy+cy^2$ et $g(x,y)=a'x^2+b'xy+c'y^2$.
\subsubsection{Dirichlet composition, class group}

Construction de la composition : grosso modo ce sera une opération qui donnera une structure de groupes a $C(D)$.
On cherche donc une manière de "composer" systématiquement deux formes primitives de discriminant $D$ pour en obtenir une nouvelle.
De la forme $f(x,y)g(z,w)=F(B_1(x,y,z,w),B_2(x,y,z,w))$ où $B_1,B_2$ sont des formes bilinéaires.\\
Bon déjà un petit rappel sur l'équivalence \textbf{propre} : (qui permet aussi d'expiquer l'algo de réduction)
\begin{description}
    \item[Diviser $b$ par $2a/2c$ :] $\begin{pmatrix}
        1 & k\\
        0 & 1
    \end{pmatrix}$ $\leftrightarrow~(x+ky,~y)$ et donne $a$, $b+2ak$, $f(1, k)$. D'où on peut réduire $b$ par la division euclidienne de $b$ par $2a$. 
    $\begin{pmatrix}
        1 & 0\\
        k & 1
    \end{pmatrix}$ Pour diviser par $c$.

    \item[Echanger $a$ et $c$ :]
    $\begin{pmatrix}
        0 & 1\\
        1 & 0
    \end{pmatrix}$ a pour déterminant $-1$ donc l'équivalence est pas propre. A la place on utilise 
    $\begin{pmatrix}
        0 & 1\\
        -1 & 0
    \end{pmatrix}$ qui donne $c$, $-b$, $a$. En particulier, $cx^2+bxy+ay^2$ est proprement équivalente à $ax^2-bxy+cy^2$.
    \item[Combinaison des deux]
    $\begin{pmatrix}
        0 & 1\\
        -1 & 0
    \end{pmatrix}$
    $\begin{pmatrix}
        1 & 0\\
        k & 1
    \end{pmatrix}=\begin{pmatrix}
        0 & 1\\
        -1 & k
    \end{pmatrix}$ et cette matrice donne $c$, $-(b+2kc)$, $f(-1,k)$. Au lieu de $k$ on met $sgn(c)k$ on écrit $-b=2\lvert c\rvert k+r$ avec $\lvert c\rvert\leq r<\lvert c\rvert$.
    Ce qui donne l'algorithme de réduction. ($\lvert-b-2k\lvert c\rvert\rvert\leq c$, si le nouveau $c$ est plus grand que le nouveau $a$ on a fini, sinon on recommence.)
\end{description}

Maintenant il convient de chercher
\begin{itemize}
    \item $B\equiv b~mod~2a$ et $B\equiv b'~mod~2a'$
\end{itemize}
de sorte qu'on peut prendre $f(x,y)=ax^2+Bxy+aCy^2$ et $g(x,y)=a'x^2+Bxy+a'Cy^2$. (Ecrire $B^2-4aC=D$ et $B^2-4a'C'=D$, de sorte que $a'C'=aC$ et on peut écrire la forme donnée.)
En écrivant $f(x,y)g(x,y)$ on trouve une forme $F(x,y)=aa'x^2+Bxy+Cy^2$ le problème c'est que le $C$ là est pas forcément entier. D'où
y faut rajouter 
\begin{itemize}
    \item $B^2\equiv D~mod~4aa'$.
\end{itemize}. Résoudre la dernière necessite des hypothèses sur $f,g$. On doit ajouter
\begin{itemize}
    \item $pgcd(a, a', \frac{b+b'}{2})=1$, car la dernière congruence, avec les deux premières équivaut à : $$\frac{b+b'}{2}B\equiv\frac{bb'+D}{2}~mod~2aa'$$
\end{itemize}
Les trois congruences sont résolues p.43-44 du Cox. 
\begin{description}
    \item[Composition de Dirichlet :] La forme obtenue est \textbf{primitive}, \textbf{définie}, \textbf{positive} de discriminant $D$.
\end{description}

L'interêt est que dans chaque deux classes il existe toujours deux formes vérifiant les hypothèses pour obtenir une composition. On obtient\\

\textbf{\gr C(D) est un groupe abélien avec la composition de Dirichlet.}

On a :
\begin{itemize}
    \item L'inverse de $ax^2+bxy+cy^2$ est $ax^2-bxy+cy^2$. (C'est là qu'on utilise $ax^2-bxy+cy^2\leftrightarrow cx^2+bxy+ay^2$ qui permet de composer les deux.)
    \item La classe principale est l'élément neutre. (Ca revient à $H$ est un sous-groupe de $(\Z/D\Z)^{\times}$. Mais la preuve est simple et directe avec Dirichlet.)
\end{itemize}
\newpage
\subsubsection{Elements of order 2 of the class group}
On va chercher la structure du groupe de classe. On commence par les élements d'ordre $\leq2$ qui sont en fait cruciaux.

\begin{description}
    \item[Une condition n et s pour être d'ordre $2$ :] $b=0$, $a=b$ ou $a=c$. (C'est clair $f=-f$ veut dire qu'elles sont proprement équiv, prendre $f$ réduite.)
\end{description}
Une conséquence : en notant,
\begin{itemize}
    \item $r$ le nb de divisieurs premiers de $D$. 
    \item Si $D\equiv 1~mod~4$, $\mu=r$.
    \item Si $D\equiv 0~mod~4$, \begin{flalign*}
        n\equiv 3~mod~4,&~\mu=r\\
        n\equiv 1,2~mod~4,&~\mu=r+1\\
        n\equiv 4~mod~8,&~\mu=r+1\\
        n\equiv 0~mod~4,&~\mu=r+2\\
    \end{flalign*}
\end{itemize} 

On a :
\begin{description}
    \item[Nombre d'élements d'ordre $2$ :] $C(D)$ a exactement $2^{\mu-1}$ éléments d'ordre $\leq2$ !
\end{description}

L'idée est d'utiliser la caractérisation des éléments d'ordre $\leq2$ et de compter le nombre de formes réduites de ce type. (p.47 du Cox) \\ \indent Par exemple pour $D\equiv0~mod~4$,
$n\equiv1~mod~4$, $b=0$. On a $n=ac$ et $2\ne n$ et $pgcd(a,c)=1$ avec $a<c$ pour être réduite. Alors on a $2^{r-1}$ choix pour $a$. (le produit des puissances max de $p_i\mid n$)\\
\newline
\indent On regarde maintenant le lien entre les élements d'ordre $2$ et le groupe de classes.

\newpage

\subsubsection{The form class group}

On considère \begin{flalign*}
    \Phi~:~C(D)&\rightarrow \textrm{ker}(\chi)/H\\
        \overline{f} &\mapsto \text{Une valeur représentée}
\end{flalign*}

Chaque fibre $\Phi^{-1}(aH)$ est l'ensemble des classes d'un genre (genus) donné, ici $a$. Pour l'instant on est pas sûrs de la surjectivité, mais 
l'image de $\Phi$ est identifiée à l'ensemble des genres (genera) possibles.

\begin{itemize}
    \item $\Phi$ est un morphisme de groupe ! (Par construction)
\end{itemize}

Un corollaire :
\begin{description}
    \item[Fibres :] Chaque genre consiste en le même nombre de classes ! (morphisme de groupe)
    \item[Nombre de genres :] Le nombre de genre de formes de discriminant D est une puissance de $2$.
\end{description}

Le second point vient du fait que $H$ contient les carrés ! D'où $(\textrm{ker}(\chi)/H)^2=\{H\}$ et $\lvert\textrm{ker}(\chi)/H\rvert=2^k$.
En fait on a un résultat bien plus précis, on redéfini :
\begin{itemize}
    \item $r$ le nb de divisieurs premiers de $D$. 
    \item Si $D\equiv 1~mod~4$, $\mu=r$.
    \item Si $D\equiv 0~mod~4$, \begin{flalign*}
        n\equiv 3~mod~4,&~\mu=r\\
        n\equiv 1,2~mod~4,&~\mu=r+1\\
        n\equiv 4~mod~8,&~\mu=r+1\\
        n\equiv 0~mod~4,&~\mu=r+2\\
    \end{flalign*}
\end{itemize} 

Et on a alors :

\begin{description}
    \item[Cardinal de $ker(\chi)/H$ :] On a $\lvert ker(\chi)/H\rvert=2^{\mu-1}$.
\end{description}

Pour le prouver, on construit un morphisme de $(\Z/D\Z)^{\times}$ dans $\{\pm 1\}^k$ surjectif et de kernel $H$. (le $k$ sera $\mu$)\\
\indent $ker(\chi)$ étant d'indice $2$ on aura fini. Le morphisme s'écrit en décomposant $(\Z/D\Z)^{\times}$ via le lemme chinois. On écrit :
\begin{flalign*}
    &&\Psi~:~&(\Z/D\Z)^{\times}\simeq \prod_i\Z/p_i^{k_i}\Z\times \Z/2^{k+2}\Z\longrightarrow \{\pm 1\}^{\mu}&& (*)
\end{flalign*}
ou chaque $\Z/p_i^{k_i}\Z$ est envoyé via $\left(\frac{.}{p_i}\right)$. ($p_i$ impair)\\
La difficulté vient du $2$. Si $D\equiv 1~mod~4$ c'est facile de montrer la surjectivité pour $k=\mu=r$, la forme principale a pour image l'ensemble des carrés mod $D$. Sinon y'a
pleins de cas à séparer pour $n\equiv i~mod~8$. Et on envoie $a~mod~D$ via $\left(\frac{2}{.}\right)$ et $\left(\frac{-1}{.}\right)$ ou leur produit. L'idée est que les éléments de $H$
s'écrivent $\beta^2, \beta^2+n~mod~D$ et il faut utiliser judicieusement les deux morphismes plus haut pour gérer les différents cas.

\begin{thm}
    On considère le morphisme 
    \begin{align*}
        \Phi~:~C(D)&\longrightarrow ker(\chi)/H\\
        f(x,y)&\mapsto \text{coset~representé}
    \end{align*}
    Alors : \\ \indent $(i)$ $\Phi$ est surjectif.\\
    \indent $(ii)$ $ker(\Phi)=C(D)^2$, autrement dit $C(D)/C(D)^2\cong ker(\chi)/H\cong \{\pm 1\}^{\mu-1}$, d'ou en particulier il y'a $2^{\mu-1}$ genres et les formes de genres principales sont toutes des carrés.
    \\ \indent 
\end{thm}

\textbf{Preuve :} On prouve $(i)$ simplement avec Dirichlet. On prouve $(ii)$ à l'aide de la suite exacte : $$0\rightarrow C(D)_2\rightarrow C(D)\rightarrow C(D)^2\rightarrow 0$$
ou $C(D)_2$ désigne les éléments d'ordre $2$, que $C(D)/C(D)^2\simeq C(D)_2$ qui a pour cardinal $2^{\mu-1}$. Comme $\Phi$ est surjective et $C(D)/C(D)^2\approx ker(\chi)/H$ comme on vient de le voir on obtient :
 $$C(D)/C(D)^2\simeq ker(\chi)/H$$.\qed

\textbf{A noter}, on a écrit un morphisme $\Phi:(\Z/D\Z)^{\times}\rightarrow \{\pm 1\}^{\mu}$ de ker $H$. D'où chaque classe d'un même genre est envoyée sur
un meme élément de $\{\pm 1\}^{\mu}$ ce qui permet de déterminer rapidement le genre d'une classe.\\

Ce qui conclut l'étude basique du groupe de classe !

\subsubsection{Convenient numbers}
Quelques résultats interessants de Euler/Gauss sur les discriminants ou chaque genre est constitué d'une unique classe :

\begin{description}
    \item[Nombre convénient :] $n$ est dit convénient si pour tout $m$ impair premier à $n$, $m=x^2+ny^2$ (proprement) n'a qu'une solution avec $x,y\geq0$ alors $m$ est premier.
    \item[Equivalence due à Gauss :] $n$ est convénient si et seulement $h(-4n)=2^{\mu-1}$ ou autrement dit, chaque genre de $C(-4n)$ consiste qu'en une classe.
\end{description}

La preuve de l'équivalence se base sur le nombre de représentations d'un nombre p.55 du Cox.

Schema à faire sur les morphismes/interdépendances

\newpage

\section{Ordres}
On définit les ordres de plusieurs manières : Un ordre $\Or\subset K$ d'un corps quadratique est :
\begin{description}
    \item[(i)] Un sous anneau de $K$
    \item[(ii)] Un sous $\Z$-module de type fini contenant une $\Q$-base de $K$.
\end{description}
Comme $\Or$ est sans torsion, (ii) revient à être un $\Z$-module libre de rang $2$.

\begin{itemize}
    \item $\OK$ est un ordre maximal. Et on a $\OK=[1, w_K]$ avec $w_K=\frac{d_K+\sqrt{d_K}}{2}$.
    \item $[\OK:\Or]=f$ est fini et $\Or=[1,fw_K]$. (même index et inclus trivialement l'un dans l'autre)
    \item On note $f$ le conducteur de $\Or$.
\end{itemize}

Avec la formule usuelle du discriminant on obtient :
\begin{itemize}
    \item $D_{\Or}=f^2d_K$.
\end{itemize}

On obtient que chaque discriminant correspond de manière unique à un ordre.

\newpage

\subsection{Idéaux et Groupe de classe d'un ordre}
Si $\ai\leq \Or$ est un idéal alors :(On peut prendre $\bi$ fractionnaire aussi.)
 
\begin{itemize}
    \item $N(\ai):=\lvert \Or/\ai\rvert$ est fini.
    \item $\Or$ est noethérien et de dimension $1$ mais clairement pas intégralement clos dès que $f>1$. D'ou
    en particulier on a pas de décomposition unique en idéaux premiers.
\end{itemize}

En fait la décomposition existe pour un sous-ensemble d'idéaux un peu plus petit que les suivants.
\begin{description}
    \item[Idéaux propres :] $\ai$ est propre ssi $\{\beta\in K:\beta\ai\subset\ai\}=\Or$. (En général on peut avoir $=\OK$)
\end{description}

En fait, on pourra former un groupe de classe grâce à au fait que 
\begin{itemize}
    \item Les idéaux propres sont exactement les idéaux inversibles. 
    \item $\implies$ se montre en remarquant que sachant $dim_{\Z}\ai=2$, $\ai=[\alpha,\beta]=\alpha[1,\tau]$ ($\mu_{\tau}=ax^2+bx+c$), alors $\Or=[1, a\tau]$. ($[1, \tau]$ est propre pour $[1, a\tau]$)
    En notant $\tau'$ le conjugué de $\tau$ et $\ai'=\alpha'[1, \tau']$. On a enfin $a\ai\ai'=a\alpha\alpha'[1,\tau][1,\tau']=N(\alpha)\Or$.
\end{itemize}

\begin{description}
    \item[Groupe de classe :] On note $C(\Or)=I(\Or)/P(\Or)$ le groupe de classe formé par les idéaux propres.
\end{description}

\newpage
\subsection{Liens avec les formes quadratiques binaires}
En fait y'a une correspondance, même un isomorphismes entre les groupes de classes. On parlera tjr de l'ordre $\Or$ de discriminant $D$.
\begin{description}
    \item[La relation :] Une forme primitive définie positive de discriminant $D$ fait naître un idéal propre de $\Or$, $[a, (-b+\sqrt{D})/2]$.
    \item[L'isomorphisme :]
    \begin{align*}
        \theta~:~C(D)&\rightarrow C(\Or)\\
        f(x,y)&\mapsto [a, (-b+\sqrt{D})/2]
    \end{align*}
    \item[Représentation/norme :] Un entier positif $m$ est représenté par une forme $f$ ssi $m$ est une norme de la classe de l'idéal correspondant dans $C(\Or)$.
\end{description}

En particulier, $h(D)=h(\Or)$. 

\subsubsection{Preuves} 
On note $\Or$ l'ordre de discriminant $D$.
\begin{itemize}
    \item \textbf{\gr La relation}, on a 
\end{itemize}
\begin{align*}
     [a, (-b+\sqrt{D})/2][a, (-b-\sqrt{D})/2] = &[a^2, a(-b+\sqrt{D})/2, -ab, ac]\\
                                              = &a[a,a(-b+f\sqrt{d_K})/2a, -b, c]\\
                                              = &a[1,fw_K]\\
\end{align*}
En remarquant que  $-b$ et $fd_K$ ont la même parité. D'ou $\theta(f(x,y)=ax^2+bxy+cy^)$ est propre car inversible.
\begin{itemize}
    \item \textbf{\gr L'isomorphisme}, qu'on prouve en deux parties. D'abord la bijection puis l'isomorphisme.
    \item Pour \textbf{\gr l'injectivité} de $\theta$, avec $\tau,\tau'$ tq
    $f(\tau,1)=g(\tau')=0$. On a

\end{itemize}
\begin{align*}
    f\sim g&\leftrightarrow \tau'=\frac{p\tau+q}{r\tau+s},\quad\begin{pmatrix}
        p & q\\
        r & s
    \end{pmatrix}\in SL_2(\Z)\\
    &\leftrightarrow [1,\tau]=\lambda[1,\tau'],\quad \lambda\in K\\
\end{align*}
En particulier $\theta(f)=\theta(g)\implies f=g$. La surjectivité se montre facilement.\\
On a pour l'instant $$C(D)\approx C(\Or)$$








\end{document}
