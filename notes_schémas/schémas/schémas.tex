\documentclass[12pt]{article}
\usepackage[dvipsnames]{xcolor}
\usepackage{hyperref, pagecolor, mdframed }
\usepackage{graphicx, amsmath, latexsym, amsfonts, amssymb, amsthm,
amscd, geometry, xspace, enumerate, mathtools}
\usepackage{tikz}
\newcommand{\F}{\mathcal{F}}
\newcommand{\Topo}{\mathfrak{T}\mathfrak{o}\mathfrak{p}}
\newcommand{\G}{\mathcal{G}}

\theoremstyle{plain}
\newtheorem{thm}[subsubsection]{Th\'eor\`eme}
\newtheorem{lem}[subsubsection]{Lemme}

\theoremstyle{definition}
\newtheorem{defn}[subsubsection]{Definition}
\newtheorem{prop}[subsubsection]{Proposition}

\definecolor{wgrey}{RGB}{50,28,77}
\hypersetup{
    colorlinks=true,
    linkcolor=blue,
    urlcolor=red
}
\title{Schémas}
\date{25 novembre 2023}
\begin{document}
\tableofcontents
\maketitle


\section{Première défs et props}
Ducoup, $X$ un e.t. et $\Topo(X)$ la catégorie des ouverts de $X$ où les morphismes sont les inclusions.

\begin{defn}
    Un prefaisceau $\F$ sur $X$ est un foncteur contravariant de $\Topo$ dans $Ab/Ring/etc.$.
\end{defn}

\begin{defn}
    Un faisceau est un préfaisceau qui satisfait
    des conditions de recollements. Etant donné $s,s'\in \F U=\cup U_i$
    \[s|_{U_i}=s|_{U_j} \implies s=s'\]
    Et si on a des sections locales qui se recollent bien 
    alors il existe une unique section globale qui les 
    relèvent.
\end{defn}

\begin{thm}
    Les faisceaux sont entièrements déterminés par les stalks, un morphisme de faisceaux 
    \[\F\rightarrow\G\]
    est un isomorphisme ssi $\forall P$ 
    \[\F_P\rightarrow \G_P\]
    est un isomorphisme.
\end{thm}
C'est aussi vrai pour les injection/surjection

\subsection{Faisceautisation}
On considère:
\begin{defn}
    \[\F^+U:=\{(s_P)\in \prod_{P\in V} \F_p | 
    \forall P~ \exists V, s_P = (\sigma, U)\in \F_P \}\]
\end{defn}
I.e. Les sections sont exactement celles qui sont relevables !
\begin{thm}
    $\F^+$ est un faisceau et il existe $\F\rightarrow \F^+$
    t.q $\forall$ faisceau $ \G$ on ait 
    \[ \F\rightarrow \F^+\rightarrow \G \]
    avec des propriétés d'unicité.
\end{thm}



\end{document}