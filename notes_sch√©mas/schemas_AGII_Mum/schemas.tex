\documentclass[12pt]{article}
\usepackage[dvipsnames]{xcolor}
\usepackage{hyperref, pagecolor, mdframed }
\usepackage{graphicx, amsmath, latexsym, amsfonts, amssymb, amsthm,
amscd, geometry, xspace, enumerate, mathtools}
\usepackage{tikz}
\newcommand{\F}{\mathcal{F}}
\newcommand{\Topo}{\mathfrak{T}\mathfrak{o}\mathfrak{p}}
\newcommand{\G}{\mathcal{G}}
\newcommand{\p}{\mathfrak{p}}
\newcommand{\K}{\mathcal{K}}
\newcommand{\A}{\mathfrak{a}}
\definecolor{wgrey}{RGB}{50,28,77}
\hypersetup{
    colorlinks=true,
    linkcolor=blue,
    urlcolor=red
}
\title{Schémas, notes 1}
\date{20 juillet 2023}
\begin{document}
\tableofcontents
\maketitle
Une petite remarque, faudra surtout pas oublier de faire des exos et écrire. Je pourrai peut être les inclure desfois.
\section{Premieres defs}
On considère que des anneaux commutatifs. Une notation de Mumford pour $$x=[\p]\in Spec(R)$$ on note $$\K(x)=R/\p$$ et 
\textbf{$f(x)=f~mod~\p\in\K(x)$}. \\

\newline \indent Bon en gros les fermés $V(S)$ c'est les premiers au dessus de $S$.
Les idéaux $I(S)$ c'est les éléments de $R$ qui s'annulent en $\p\in S$ sur tout $S$.
On a le nullstellensatz les propriétés usuelles, etc, etc..


\end{document}