\documentclass[12pt]{article}
\usepackage[dvipsnames]{xcolor}
\usepackage{hyperref, pagecolor, mdframed }
\usepackage{graphicx, amsmath, latexsym, amsfonts, amssymb, amsthm,
amscd, geometry, xspace, enumerate, mathtools}
\usepackage{tikz}

\theoremstyle{plain}
\newtheorem{thm}[subsubsection]{Th\'eor\`eme}
\newtheorem{lem}[subsubsection]{Lemme}
\newtheorem{prop}[subsubsection]{Proposition}
\newtheorem{propr}[subsubsection]{Propri\'et\'e}
\newtheorem{cor}[subsubsection]{Corollaire}
\newtheorem{intro}[section]{Introduction}
\newtheorem{thm2}[subsection]{Th\'eor\`eme}
\newtheorem{lem2}[subsection]{Lemme}
\newtheorem{prop2}[subsection]{Proposition}
\newtheorem{propr2}[subsection]{Propri\'et\'e}
\newtheorem{cor2}[subsection]{Corollaire}

\theoremstyle{definition}
\newtheorem{defn}[subsubsection]{D\'efinition}
\newtheorem{rmq}[subsubsection]{Remarque}
\newtheorem{conj}[subsubsection]{Conjecture}
\newtheorem{exmp}[subsubsection]{Exemples}
\newtheorem{quest}[subsubsection]{Exercices}
\newtheorem{defn2}[subsection]{D\'efinition}
\newtheorem{rmq2}[subsection]{Remarque}
\newtheorem{conj2}[subsection]{Conjecture}
\newtheorem{exmp2}[subsection]{Exemples}
\newtheorem{quest2}[subsection]{Exercices}

\theoremstyle{remark}
\newtheorem{rem}{Remarque}
\newtheorem{note}{Note}

\newcommand{\fdiv}{\textrm{div}}
\newcommand{\N}{\mathbb{N}}
\newcommand{\Z}{\mathbb{Z}}
\newcommand{\Q}{\mathbb{Q}}
\newcommand{\K}{\mathbb{K}}
\newcommand{\Proj}{\mathbb{P}}
\newcommand{\algK}{\overline{K}}
\newcommand{\algF}{\overline{\mathbb{F}}}
\newcommand{\Pic}{\textrm{Pic}}
\newcommand{\Hom}{\textrm{Hom}}
\newcommand{\End}{\textrm{End}}
\newcommand{\C}{\mathbb{C}}
\newcommand{\w}{\omega}
\newcommand{\h}{\mathfrak{h}}
\newcommand{\La}{\mathcal{L}}
\newcommand{\D}{\mathcal{D}}
\newcommand{\F}{\mathcal{F}}
\hypersetup{
    colorlinks=true,
    linkcolor=blue,
    urlcolor=Green,
    filecolor=RoyalPurple
}

\definecolor{wgrey}{RGB}{148, 38, 55}
\title{Formes modulaires}
\date{13 aout 2023}
\begin{document}
\tableofcontents
\maketitle

En suivant "A first course in modular forms".
\section{Structure de surface de Riemann des courbes modulaires}
\subsection{Sous-groupes de congruences}
On note $\pi_N$ la projection $SL_2(\Z)\rightarrow SL_2(\Z/N\Z)$.
\begin{defn}
Les sous groupes principaux :
\begin{enumerate}
    \item $\Gamma(N):=\{\gamma\in SL_2(\Z);~\pi_N(\gamma)=I\}$
    \item $\Gamma_0(N):=\{\gamma\in SL_2(\Z);~\pi_N(\gamma)=\begin{pmatrix}*&*\\ 0&* \end{pmatrix}\}$
    \item $\Gamma_1(N):=\{\gamma\in SL_2(\Z);~\pi_N(\gamma)=I+\begin{pmatrix}0&*\\ 0&0 \end{pmatrix}\}$
\end{enumerate}
On a $\Gamma(N)\subset \Gamma_1(N)\subset \Gamma_0(N)$. 
\end{defn}

\begin{defn}
    On définit $Y(N),Y_0(N),Y_1(N)$ comme $\Gamma(N)\backslash\h$, $\Gamma_0(N)\backslash\h$, $\Gamma_1(N)\backslash\h$. 
    Les courbes modulaires (auxquelles il manque des points).
\end{defn}

\begin{rem}
    Ducoup c'est des moduli space de courbes elliptiques + torsion. En particulier : $Y_0(N)\backslash \h$ a pour point 
    les classes d'équivalence sur $\{(E,G);~\text{G est d'ordre N}\}$ pour la relation : $$(E,G)\sim(E',G')$$ ssi il existe une isogénie qui envoie 
    $G$ sur $G'$. En particulier, $(E,G)\sim (E/G, \{O\})$. (Voir 1.5. du livre)
\end{rem}

Pour $\Gamma(N)\subset\Gamma\subset SL_2(\Z)$ on définit de meme $Y(\Gamma)$.

\subsection{Topologie de $Y(\Gamma)$.}
On utilise la topologie quotient via la projection $\pi:\h\rightarrow Y(\Gamma)$, alors :
\begin{enumerate}
    \item $\pi(U_1)\cap\pi(U_2)=\emptyset$ si et seulement si $\Gamma U_1\cap U_2=\emptyset$
    \item En plus, on peut trouver des ouverts suffisamment petits $\tau_1\in U$, $\tau_2\in V$ tels que
     $$\forall\gamma\quad \gamma U\cap V\ne\emptyset \implies \gamma(\tau_1)=\tau_2$$En fait pendant la preuve on montre aussi 
     que $\{\gamma;~\gamma U_1\cap U_2\ne\emptyset\}$ est fini. (En utilisant $Im(\gamma\tau)=Im(\tau)/\lvert c\tau+d\rvert$, 
     remarque que la partie imaginaire a tendance a diminuer et pas grandir. Ensuite on moyenne les ouverts obtenus)
    \item On utilise ces ouverts pour montrer que la topologie est Hausdorff. (On compactifie après.)
\end{enumerate}

\subsection{Cartes et points elliptiques}
On regarde $i:\Gamma\subset SL_2(\Z)\xhookrightarrow{PSL_2} SL_2(\Z)/\{\pm1\}$.($\{\pm1\}\Gamma/\{\pm1\}$) :
\begin{defn}
    Sous-groupe d'\textbf{isotropie} : $\Gamma_{\tau}:=\{\gamma\in \Gamma;~\gamma\tau=\tau\}$. \\Et \textbf{periode} de $\tau$ :
    $h_{\tau}:=\begin{cases}
        \lvert \Gamma_{\tau}/2\rvert &si~-I\in\Gamma_{\tau}\\
        \lvert \Gamma_{\tau}\rvert &sinon
    \end{cases}$, autrement dit $h_{\tau}=\lvert i(\Gamma_{\tau})\rvert$.
\end{defn}

La periode est définissable car le sous groupe d'isotropie est fini. (A voir après) 

\begin{abstract}
    On cherche maintenant les cartes et coordonnées locales : la periode est définie sur $Y(\Gamma)$ lorsque $\Gamma$ est distingué et on regarde 
    l'image dans $PSL_2(\Z)$ car $-I$ agit toujours trivialement sur $Y(\Gamma)$. Maintenant les étapes, en gros les points problématiques
    c'est les points elliptiques psq les autres $\pi$ est localement injective, ducoup on regarde un petit ouvert d'un point elliptique intersecté
    avec un domaine fondamental (voir un dessin) :
    \begin{enumerate}
        \item On se ramène à $0$ via $\delta_{\tau}:=\begin{pmatrix} 1 &-\tau\\ 1 & -\overline{\tau}\end{pmatrix}$
        \item On remarque que les conjugués $\delta_{\tau}\Gamma_{\tau}\delta_{\tau}^{-1}$ fixent $0,\infty$ et 
        étant des homographies sont linéaires. Enfin par le point d'avant c'est de cardinal $h_{\tau}$ en tant que 
        groupe de transformations (dans $PSL_2$).
        \item Ce sont donc des rotations d'angle $2\pi/h_{\tau}$. La on peut visualiser : $\delta_{\tau}$ envoie donc 
        un petit voisinage de $\tau$ sur une part de cercle (littéralement) de pointe $0$. On obtient une boule en mettant a la puissance
        $h_{\tau}$.
        \item Ensuite, il existe $\tau\in U$ tq pour tout $\gamma$, $\gamma U\cap U\ne\emptyset$ implique que $\gamma\in\Gamma_{\tau}$.
        \item D'ou on prend $\overline{U}:=\Gamma_{\tau}\backslash U$ et $\delta_{\tau}^{h_{\tau}}$ comme coordonnée locale.
    \end{enumerate}
\end{abstract}

Pour les points elliptiques, on remarque plusieurs choses :
\begin{enumerate}
    \item topologiquement $Y(1)$ est un plan et a pour domaine fonda : $\D:=\{z;~\lvert z\rvert\geq 1,~\lvert Re(z)\rvert \leq 1/2\}$
    \item Les points de $\D$ qui restent dans $\D$ après une transfo sont au bord.
    \item Les points elliptiques : écrire le disc de $a\tau+b=c\tau^2+d\tau$ donne $\lvert a+d\rvert<2$ puis
    le pol caractéristique de $\gamma$ s'écrit $x^2+(a+d)X+1$, d'ou $\gamma^6=I$ et y'a une jolie preuve pour préciser ca dans le livre.
    \item Les points elliptiques pour $SL_2(\Z)$ sont $SL_2(\Z).i$ et $SL_2(\Z).\mu_3$ de groupes $<S>=<\begin{pmatrix}0&-1\\1&0 \end{pmatrix}$
    et $<ST>=<\begin{pmatrix}0&-1\\ 1&1 \end{pmatrix}$. Groupes finis cycliques.
\end{enumerate}

Enfin comme $SL_2(\Z)=\cup_{i=1}^{d}\Gamma\gamma_i$ d'indice fini:
\begin{prop}
    Les points elliptiques de $\Gamma$ sont contenus dans $\Gamma.\{\gamma_j(i),\gamma_j(\mu_3);~j\}$. Donc nombre fini
    et les groupes d'isotropie sont finis cycliques aussi.
\end{prop}

\subsection{pointes}
On compactifie maintenant $\h$ d'une certaine manière : $\h^*:=\h\cup\Proj^1(\Q)$. 
\begin{defn}
    $X(\Gamma):=\Gamma\backslash \h^*$ ou l'action de $\Gamma$ sur $\Proj^1(\Q)$ est l'action par homographies.
\end{defn}

Quand $\Gamma\ne SL_2(\Z)$ y'a plus de pointes que $\infty$ ducoup prendre des $\{\lvert z\rvert > r\}\cap \h^*$ 
ca contient trop de points de $\Proj^1(\Q)$ en gros en quotientant, tente de séparer deux pointes $\ne\infty$.

\begin{defn}
    On rajoute aux ouvert de $\h$ les boules $N_m:=\{Im(\tau)> m\}$ et les images $\alpha(N_m)$ ou 
    $\alpha$ envoie $\infty$ sur $q\in\Q$.
\end{defn}
Les transformations sont conformes, contenues dans $\h\cup\Q$ d'ou des disques tangent à $\mathbb{R}$.


\begin{prop}
    $X(\Gamma)$ est Hausdorff, connexe et compacte.
\end{prop}

Y'a quelques étapes en plus pour la structure de surface de Riemann par rapport à $X(1)$. C'est p.59 (a regarder)




\end{document}

\section{formes modulaires et dimension}
