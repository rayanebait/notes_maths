\documentclass[12pt]{article}
\usepackage[dvipsnames]{xcolor}
\usepackage{hyperref, pagecolor, mdframed }
\usepackage{graphicx, amsmath, latexsym, amsfonts, amssymb, amsthm,
amscd, geometry, xspace, enumerate, mathtools}

\hypersetup{
    colorlinks=true,
    linkcolor=blue,
    urlcolor=red
}
\title{SEA, notes 1}
\date{30 juin 2023}
\begin{document}
\tableofcontents
\maketitle
\noindent Ref 1 : \href{https://scholar.harvard.edu/files/kieffer/files/sea.pdf}{mémoire de J.Kieffer}.\\
Ref 2 : \href{https://www.mat.uniroma2.it/~schoof/ctg.pdf}{Papier de Schoof}.

\section{Defs de la ref 1}
\indent \indent {\color{RoyalPurple}Polynome de noyau} de $G\leq E$ : $\prod_{P\in G}(X-x(P))$
\section{Sur la partie II de la ref 1}
\indent \indent {\textbf{\color{Sepia}Page 6}} : Isogénie de degré $l$ premier à $p$. On a 
\begin{align*}
    \phi~:~E&\to E'\\
    (x,y)&\mapsto (\phi_x(x,y), cy\phi_x'(x,y))
\end{align*}

\noindent Ecrire $\frac{dx'\circ\phi}{y'\circ\phi}=\frac{d\phi_x}{\phi_y}$ en fonction de $\frac{dx}{y}$.
On peut surement appliquer R.R et obtenir une relation $\overline{\mathbb{F}}_q$-dépendante.
\\ \newline
\indent {\textbf{\color{Sepia}Page 8}} : A la fin, la deuxième équivalence est claire en écrivant $\phi=$ 
$\begin{pmatrix} 
    \alpha & \gamma\\
    0 & \beta
\end{pmatrix}$
via l'espace propre $E_{\alpha}$. Pour la première équivalence On
note $t^2-4q=w^2~mod~l$ alors $q=2^{-1}(t-w)2^{-1}(t+w)~mod~l$ i.e. $\alpha=2^{-1}(t-w)$
et $\beta=2^{-1}(t+w)$ d'ou $q=\alpha\beta$ et $t=\alpha+\beta$. Faut vérif que $<\alpha>$ est rationnel.
Inversement 


\section{Sur la partie 6 de la ref 2}

\end{document}